%% start of file 'cv.tex'.
\documentclass[11pt,a4paper]{moderncv}

% moderncv themes
%\moderncvtheme[blue]{casual}                 % optional argument are 'blue' (default), 'orange', 'red', 'green', 'grey' and 'roman' (for roman fonts, instead of sans serif fonts)
\moderncvtheme[red]{classic}                 % idem

% character encoding
\usepackage[utf8]{inputenc}                   % replace by the encoding you are using

% adjust the page margins
\usepackage[scale=0.84]{geometry}
\setlength{\hintscolumnwidth}{2.6cm}			% if you want to change the width of the column with the dates
%\AtBeginDocument{\setlength{\maketitlenamewidth}{6cm}}   % only for the classic theme, if you want to change the width of your name placeholder (to leave more space for your address details
\AtBeginDocument{\recomputelengths}                       % required when changes are made to page layout lengths

% personal data
\firstname{Maxime}
\familyname{Beugnet}
\title{Ingénieur en Informatique: Génie Logiciel\newline\mbox{~~~~~~~~~~~~4 années d'expérience}} % optional, remove the line if not wanted
%\title{\centerline{Ingénieur en Informatique: Génie Logiciel}\newline\centerline{2 années~d'expérience}} % optional, remove the line if not wanted
%\address{15 rue Victor Hugo}{92400 Courbevoie}	% optional, remove the line if not wanted
%\mobile{+33 7 81 26 42 16}                              % optional, remove the line if not wanted
%\phone{phone (optional)}                               % optional, remove the line if not wanted
%\fax{fax (optional)}                                   % optional, remove the line if not wanted
%\email{maxime.beugnet@zenika.com}                        % optional, remove the line if not wanted
%\extrainfo{26 ans, né à Toulouse\\Français, célibataire, sans enfant\\Permis de conduire B} % optional, remove the line if not wanted

%\photo[64pt]{QR_code.png}     % '64pt' is the height the picture must be resized to and 'picture' is the name of the picture file; optional, remove the line if not wanted
\photo[200pt]{zenika.jpg}     % '64pt' is the height the picture must be resized to and 'picture' is the name of the picture file; optional, remove the line if not wanted
\quote{Développeur passionné de nouvelles technologies.}    % optional, remove the line if not wanted

%\nopagenumbers{}                             % uncomment to suppress automatic page numbering for CVs longer than one page


%----------------------------------------------------------------------------------
%            content
%----------------------------------------------------------------------------------
\begin{document}
\maketitle

\section{Expériences Professionnelles}
%\subsection{Vocational}

\cventry
{Mars\ 2015/\\Octobre 2015}
{Consultant Zenika - Développeur Java/JEE et AngularJS}
{BNP Paribas}
{Paris}
{}
{\begin{itemize}
\item Développement sur un projet de design d'architecture logiciel,
\item Autonomie complète sur le projet, 
\item Augmentation des possibilités offertes par Eclipse RCP,
\item Mise en place d'une intégration continue avec Vagrant et Atlassian.
\end{itemize}
$\rightarrow$ Compétences utilisées: IntelliJ, Java/JEE, AngularJS, Spring, Hibernate, MySQL, Jira, Stash, Git, Jenkins, Bamboo.
}   % arguments 3 to 6 are optional

\cventry
{10-12 \& 16-19\\ Juin 2015}
{Consultant Zenika - Formateur MongoDB}
{ATOS Worldline}
{Seclin et Lyon}
{}
{\begin{itemize}
\item Formation MongoDB Essentials (3 jours),
\item Formateur officiel, validé par MongoDB Inc.
\end{itemize}
$\rightarrow$ Compétences utilisées: MongoDB.
}   % arguments 3 to 6 are optional

\cventry
{28 Mai\ 2015}
{Consultant Zenika - Audit MongoDB}
{Réputation VIP}
{Lyon}
{}
{\begin{itemize}
\item \'Etude du modèle de données existant et du besoin client,
\item Refonte complète du modèle de données,
\item Optimisation des indexes et des requêtes cruciales,
\item \'Etudes des logs pour résoudre les problèmes d'instabilité du cluster,
\item Conseils pratiques et réduction global des coûts sur AWS.
\end{itemize}
$\rightarrow$ Compétences utilisées: MongoDB.
}   % arguments 3 to 6 are optional

\cventry
{Février\ 2015/\\Mars 2015}
{Consultant Zenika - Développeur Node.js et MongoDB}
{AXA}
{Paris et Bruxelles}
{}
{\begin{itemize}
\item Application de filtrage des clients pour des campagnes Marketing,
\item Découverte et formalisation du besoin client,
\item Mise en place de l'architecture logicielle,
\item Design de la base de données MongoDB,
\item Création d'une API REST Node.js pour exposer les données de MongoDB,
\item Mise en place d'Elastic pour répondre aux problématiques de performances,
\item Réalisation d'un générateur aléatoire de clients en Java 8.
\end{itemize}
$\rightarrow$ Compétences utilisées: Java 8, Node.js, MongoDB, Elastic, Docker, Codeship, Heroku, Kanban.
}   % arguments 3 to 6 are optional

\cventry
{Novembre\ 2013/\\Janvier 2015}
{Consultant Zenika - Développeur Java/JEE}
{Galec - E.Leclerc}
{Ivry-Sur-Seine}
{}
{\begin{itemize}
\item Mises à jour et évolutions en Java/JEE sur l'application ISIS-Offre,
\item Application interne critique de gestion des produits de E.Leclerc,
\item Analyse et chiffrage des correctifs et évolutions,
\item Maintenance et création d'exports PDF avec Jasper,
\item Montée en compétences sur la méthode Kanban,
\item Acteur actif du cycle de vie d'une application en TMA,
\item Formateur pour les nouveaux arrivants sur le projet.
\end{itemize}
$\rightarrow$ Compétences utilisées: Java/JEE, Hibernate, Spring, Struts, JUnit, Eclipse, PowerAMC, Jasper, CVS, Kanban.
}   % arguments 3 to 6 are optional

\cventry
{Février\ 2013/\\Novembre 2013}
{Formations et développement personnel}{}{}{}
{\begin{itemize}
\item Approfondissement de mes connaissances en Java,
\item Participation à plusieurs concours d'algorithmique,
\item Remise à niveau en JEE et approfondissement des bases acquises durant mes études,
\item Réalisation de projets personnels,
\item Veille technologique.
\end{itemize}
}   % arguments 3 to 6 are optional

\cventry
{Octobre\ 2011/\\Février 2013}
{Ingénieur Logiciel Embarqué}
{EADS Astrium}
{Stevenage (Royaume-Uni)}
{}
{\begin{itemize}
\item Développement du code embarqué en C pour le projet Lisa PathFinder,
\item Responsable des tests unitaires et des tests en boucle ouverte,
\item Analyste fonctionnel,
\item Acteur actif du cycle de vie du logiciel embarqué,
\item Mise en place, maintenance et optimisation de l'environnement de développement,
\item Création de scripts d'automatisation de tâches redondantes lourdes,
\item Optimisation des performances des différents tests et création d'outils de génération de rapports,
\item Revue de code (relecture croisée) et analyse de la qualité du code,
\item Responsable de la formation des nouveaux développeurs.
\end{itemize}
$\rightarrow$ Compétences utilisées: GNU/Linux, C, MISRA, RTEMS, Java, Script Shell, Makefile, Eclipse, ClearCase, ClearQuest, Cantata++, VectorCast Cover, Rulechecker, Logiscope, Doors.
}   % arguments 3 to 6 are optional

\cventry
{Mai 2011/\\Octobre 2011}
{Ingénieur Logiciel Embarqué - Stage $\mathbf{3^{e}}$ année}
{EADS Astrium}{Stevenage (Royaume-Uni)}
{}
{\begin{itemize}
\item Aide à la reprise du projet (arrêt sous-traitance),
\item Mise en place d'une nouvelle stratégie de Tests Unitaires,
\item Conception d'une librairie de Tests Unitaires pour remplacer Cantata++ (rétroingénierie).
\end{itemize}
$\rightarrow$ Compétences utilisées: GNU/Linux, C, Java, Script Shell, Makefile, Eclipse, ClearCase, Cantata++, VectorCast Cover.
}   % arguments 3 to 6 are optional

\cventry
{Juin 2010/\\Octobre 2010}
{Développeur - Stage $\mathbf{2^{de}}$ année}
{Capgemini}
{Pau (64)}
{}
{\begin{itemize}
\item Refonte complète de la charte graphique et intégration du CSS,
\item Réécriture de \emph{Gesper} de Visual Basic (client lourd) en C\# et .NET (client léger) pour \emph{Total}.
\item Mise en place d'importation massive de données depuis la BDD oracle vers Excel.
\end{itemize}
$\rightarrow$ Compétences utilisées: C\#, .NET, HTML, CSS, PL/SQL.
}   % arguments 3 to 6 are optional


\section{Formations}

\cventry{Avril 2015}
{Certification officielle MongoDB Administrateur}
{MongoDB, Inc}
{}
{}
{https://university.mongodb.com/exams/verify\_certificate - numéro : 497-369-558}

\cventry{Janvier 2015}
{Certification officielle MongoDB Développeur}
{MongoDB, Inc}
{}
{}
{https://university.mongodb.com/exams/verify\_certificate - numéro : 187-063-954}

\cventry{Juillet 2014}
{Formation Hadoop}
{Zenika via Pivotal}
{}
{}
{Formation de 4 jours.}

\cventry{Juin 2014}
{Formation Hibernate}
{Zenika}
{}
{}
{Formation de 2 jours.}

\cventry{Mars--Décembre 2014}
{Formation MongoDB}
{Développement personnel}
{}
{}
{Formation M101J, M102, M202 du site https://university.mongodb.com/}

\cventry{2008--2011}
{Diplôme d'Ingénieur en Informatique}
{\'Ecole Internationale des Sciences du Traitement de l’Information (EISTI)}
{Pau (64) et Cergy-Pontoise (95)}
{}
{Option Génie Logiciel.}

\cventry{2006--2008}
{Classes préparatoires aux grandes écoles}
{Lycée Saliège}
{Balma (31)}
{}
{Math Sup/Spé option Physique \& Chimie.}

\cventry{2006}
{Baccalauréat Scientifique}
{Lycée}
{Saint-Orens de Gameville (31)}
{Mention AB}
{Option mathématiques spéciales.}

\section{Compétences}

\subsection{Informatiques}
\cvline{Java/JEE}{Pratique constante depuis 2009.} 
\cvline{Frameworks}{Hibernate, Spring, Struts, XStream, Mockito, Junit, AssertJ, Apache Commons.}
\cvline{BDD}{MongoDB, Elastic, MySQL, PostgreSQL, DB2, PL/SQL.}
\cvline{BigData}{Hadoop.}
\cvline{Autres langages}{C, C\#, Script Shell, HTML5, CSS3, Javascript.}
\cvline{Systèmes}{GNU/Linux.}
\cvline{Outils}{IntelliJ, Eclipse, Docker, Vagrant, Tomcat, Maven, Git, ClearCase, ClearTool, Subversion, Bazaar, CVS, VSS.}
\cvline{Méthodologies}{Craftmanship, TDD, Kanban.}
\cvline{Standards}{ECSS, MISRA, W3C.}

\subsection{Langues}
\cvlanguage{Anglais}{Bilingue}{}
\cvlanguage{Espagnol}{Scolaire}{}

%\section{Master thesis}
%\cvline{title}{\emph{Title}}
%\cvline{supervisors}{Supervisors}
%\cvline{description}{\small Short thesis abstract}

%\newpage{}

\section{Activité Meetup \& Conférences}
\cvlistdoubleitem{Paris JUG}{Devoxx 2014 \& 2015}
\cvlistdoubleitem{Software Craftmanship Paris}{Docker Paris}
\cvlistdoubleitem{Paris MongoDB User Group}{Global Day of CodeRetreat}
\cvlistdoubleitem{Dev4Fun}{}
%\cvlistitem[+]{Item 3}            % optional other symbol

\section{Compétitions de programmation}
%\cvline{hobby 1}{\small Description}
%\cvline{hobby 2}{\small Description}
%\cvline{hobby 3}{\small Description}

\cventry
{Depuis 2013}
{CodinGame}
{}
{\url{http://www.codingame.com}}
{}
{Participation mensuelle aux compétitions avec Java.}

\cventry
{Depuis 2012}
{Google Jam Code}
{}
{\url{http://code.google.com/codejam}}
{}
{Concours annuel mondial de résolution de problèmes algorithmiques; utilisation de Java.}

%\cventry
%{Septembre 2013}
%{CodinGame}
%{}
%{\url{http://www.codingame.com}}
%{}
%{Concours de résolution de problèmes algorithmiques; utilisation de Java.\\Résultat final : $226^{e}$ sur 1124 participants.}

%\cventry
%{Juillet 2013}
%{CodinGame}
%{}
%{\url{http://www.codingame.com}}
%{}
%{Concours de résolution de problèmes algorithmiques; utilisation de Java.\\Résultat final : $128^{e}$ sur 759 participants.}

%\cventry
%{Avril/Mai 2013}
%{Google Jam Code}
%{}
%{\url{http://code.google.com/codejam}}
%{}
%{Concours mondial de résolution de problèmes algorithmiques; utilisation de Java.\\Résultat final : $5319^{e}$ sur 21278 participants.}

\cventry
{Août 2012}
{Code of Duty 2}
{}
{\url{http://labs.criteo.com/code-of-duty-2/}}
{}
{Concours de résolution de problèmes algorithmiques; utilisation de Java.\\Résultat final : Dans les 100 premiers parmi plus de 2100 participants.}

%\cventry
%{Avril/Mai 2012}
%{Google Jam Code}
%{}
%{\url{http://code.google.com/codejam}}
%{}
%{Concours mondial de résolution de problèmes algorithmiques; utilisation de Java.\\Résultat final : $1876^{e}$ sur 19464 participants.}

\cventry
{Janvier 2012}
{Queue ICPC Challenge}
{}
{\url{http://queue.acm.org/icpc/}}
{}
{Tournoi d'Intelligences Artificielles; utilisation de Java\\Résultat final : $13^{e}$ sur 53 participants.}

\renewcommand{\listitemsymbol}{-} % change the symbol for lists

%\section{Intérêts}
%\cvlistdoubleitem{Tir à l'arc}{Cinéma}
%\cvlistdoubleitem{Natation}{Musique}
%\cvlistdoubleitem{Squash}{Rubik's Cube}

% Publications from a BibTeX file
\nocite{*}
\bibliographystyle{plain}
\bibliography{publications}       % 'publications' is the name of a BibTeX file

\end{document}

